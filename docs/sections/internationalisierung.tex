\section{Internationalisierung der Anwendung}
Mit der Internationalisierung einer Anwendung kann die Anwendung ohne Änderung des Quelltextes um beliebige Sprachen erweitert werden.
Grundsätzlich gibt es in Java zwei Möglichkeiten eine Anwendung zu internatialisieren:
entweder werden die entsprechenden Texte in Resourcen-Dateien ausgelagert und diese wiederrum von der Anwendung geladen, um die Texte je nach Sprache
angezeigt zu bekommen oder es werden separate Java-Klassen geschrieben, die der Anwendung ein Array von Objekten zurückliefert.
In unserer Anwendung wurden beide Wege realisiert, jedoch ist die zweite Methode über die Java-Klassen aktiv.
Folgend ist der Inhalt der entsprechenden Dateien für die deutsche Sprache zu sehen.

\begin{lstlisting}[frame=single,language=JAVA,caption=Java Resourcen-Klasse]
    package dev.swt.gui;  
    import java.util.ListResourceBundle;     
    /**
        * German Messages
        */
    public class MessageBundle_de_DE extends ListResourceBundle {
        @Override
        protected Object[][] getContents() {
            return contents;
        }  
        private Object[][] contents = { { "editorTitle", "Text-Editor" },
            { "editorTabText", "unbenannt" }, { "editorFile", "&Datei" },
            { "editorEdit", "&Editieren" }, { "editorHelp", "&Hilfe" },
            { "editorNew", "&Neu" }, { "editorOpen", "&Oeffnen" },
            { "editorSave", "&Speichern..." }, { "editorQuit", "&Beenden" },
            { "editorColor", "Text &Farbe" }, { "editorVersion", "&Version" },
            { "colorTitle", "Farb-Auswahl" }, { "colorInfo", "Bitte waehlen Sie eine Textfarbe aus" },
            { "colorGroup", "Textfarbe" }, { "colorRed", "rot" },
            { "colorGreen", "gruen" }, { "colorBlue", "blau" },
            { "colorOkay", "Ok" }, { "colorCancel", "Abbruch" },
            { "version", "Version 1.3" }, { "owner", "Marius Schenzle" },
            { "stopClosing", "Modifizierte Tabs entdeckt.\nWollen Sie trotzdem schliessen?" } };
    }
\end{lstlisting}

\newpage

\lstinputlisting[frame=single,caption=Java Resourcen-Datei]{files/MessageBundle_de_DE.properties}

