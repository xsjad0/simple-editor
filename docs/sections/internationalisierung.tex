\section{Internationalisierung der Anwendung}
Mit der Internationalisierung einer Anwendung kann die Anwendung ohne Änderung des Quelltextes um beliebige Sprachen erweitert werden.
Grundsätzlich gibt es in Java zwei Möglichkeiten eine Anwendung zu internatialisieren:
entweder werden die entsprechenden Texte in Resourcen-Dateien ausgelagert und diese wiederrum von der Anwendung geladen, um die Texte je nach Sprache
angezeigt zu bekommen oder es werden separate Java-Klassen geschrieben, die der Anwendung ein Array von Objekten zurückliefert.
In unserer Anwendung wurden beide Wege realisiert, jedoch ist die zweite Methode über die Java-Klassen aktiv.
Folgend ist der Inhalt der entsprechenden Dateien für die deutsche Sprache zu sehen.